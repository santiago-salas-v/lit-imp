\documentclass[onecolumn]{article}
\usepackage{amsmath}
\begin{document}
\title{03 - Debye-Hueckel limiting law model}
\author{}
\date{}
\maketitle
Activity coefficient expressions - Convention III, Debye-Hueckel
limiting law on electrolytes,
Setschenow non-electrolytes, ideal osmotic coefficient $\phi=1$ in solvent:
\begin{equation}
\label{eq:activity_coeff}
\begin{aligned}
\textnormal{electrolytes: } & log_{10}(\gamma_{i}) =
-A_m \cdot z_i^2 \times \sqrt{I} \\
& A_m = A \rho_0^{1/2}/ln(10) \\
& A = (2 \pi N_{Av})^{1/2}
\left({e^2}\over{4 \pi \varepsilon_0 \varepsilon_r  k T} \right)^{3/2}\\
& H_2O, 298.15K, \varepsilon_r=78.54, \rho_0=0.99714kg/L: \\
& A/ln(10)=0.5099 (mol/L)^{-1/2}, A_m=0.5092 (mol/kg)^{-1/2} \\
\textnormal{non-electrolytes: } & log_{10}(\gamma_{i}) = b_i \times I \\
\textnormal{solvent: } & log_{10}(\gamma_{0}) =
-\phi \times M_0 \sum_{j\neq0}{m_j} \times log_{10}(e)\\
 & \phi_{id.} =   1.0 \\
\end{aligned}
\end{equation}
Ionic strength:
\begin{equation}
\label{eq:ionic_strength}
\begin{aligned}
& I = \frac{1}{2} \sum_{i=1}^n m_i z_i^2 \\
\end{aligned}
\end{equation}
Material balances:
\begin{equation}
\begin{aligned}
\label{eq:mole_balance}
n_i &= n_{0,i} + \sum_j(\nu_{ij} \times \xi_j) \\
m_i &= n_i /(n_0 M_0) \\
x_i &= \frac{n_i}{\sum_j{n_j}}= \frac{m_i M_0}{1+\sum_{j\neq0}{m_j M_0}}
\end{aligned}
\end{equation}
Equilibrium constant expressions:
\begin{equation}
\label{eq:eq_expr}
K_j(T) = \left[ \prod_i(\gamma^{II}_i x_i)^{\nu_{ij}} \right]_{\textnormal{eq.}} =
\left[ \prod_i(\gamma^{III}_i \times m_i/m^{\circ}_i )^{\nu_{ij}} \right]_{\textnormal{eq.}}
\end{equation}
Charge balance:
\begin{equation}
0 = \sum_i(z_i \times m_i)
\end{equation}
\\
Vars: \\
n+1 components composition $n_0$, $n_1$, ..., $n_n$. \\
n+1 components activity coefficients $\gamma_0$, $\gamma_1$, ..., $\gamma_n$. \\
$n_r$ reaction extents $\xi_1$, $\xi_2$, ..., $\xi_{nr}$. \\
Tot. $2n + 2 + n_r$ \\
Eqs: \\
n+1 mole balances (eq. \ref{eq:mole_balance}). \\
n+1 activity coefficient expressions (eq. \ref{eq:activity_coeff}). \\
$n_r$ equilibrium expresisons (eq. \ref{eq:eq_expr}). \\
Tot. $2n + 2 + n_r$
\end{document}

\documentclass[twocolumn]{article}
\usepackage{amsmath}
\begin{document}
\section{Conventions} % (fold)
\label{sec:section_name}
% section section_name (end)
Convention II: Ideal solvent towards unit mole fraction; ideal solutes at \textit{infinite dilution}, expressed in mole fraction. \cite{Denbigh1968} \\
\begin{equation}
\begin{aligned}
\textnormal{solvent (0): } & (\gamma_0 \rightarrow 1) as  (x_0 \rightarrow 1)\\
\textnormal{solute (i): } & (\gamma_i \rightarrow 1) as (x_i \rightarrow 0)
\end{aligned}
\end{equation}
Convention III: Ideal solvent towards unit mole fraction; ideal solutes at \textit{infinite dilution}, expressed in molality. \\
\begin{equation}
\begin{aligned}
\textnormal{solvent (0): } & (\gamma_0 \rightarrow 1) as  (x_0 \rightarrow 1)\\
\textnormal{solute (i): } & (\gamma_i \rightarrow 1) as (m_i \rightarrow 0)
\end{aligned}
\end{equation}
Chemical potential expressed under each convention, where $m_i^{\circ}$ is a reference solute state at unit molality.
\begin{equation}
\begin{aligned}
\mu_i = & \mu_i^{*} + RTln(\gamma_i^{II} x_i) \\
	=	& \mu_i^{\fbox{}} + RTln(\gamma_i^{III} m_i/m_i^{\circ})
\end{aligned}
\end{equation}
Notes on conversion between conventions: 
\begin{itemize}
\item Values of activity coefficients $\gamma_i^{II}$ and $\gamma_i^{III}$ vary, chemical potential does not.
\item Reference states chemical potentials under each convention $\mu_i^{*}$ and $\mu_i^{\fbox{}}$  are functions of temperature and pressure exclusively. Their difference calculated at infinite dilution does not change at other compositions:
\end{itemize}
\[
\mu_i^{*} - \mu_i^{\fbox{}}  =  func(T,P) =  \lim_{x_0 \to 1} \left( \mu_i^{*} - \mu_i^{\fbox{}}\right) 
\]
\[
RTln\left[ \frac{\gamma_i^{III} m_i/m_i^{\circ}}{\gamma_i^{II} x_i}\right] = \lim_{x_0 \to 1} RTln\left[ \frac{\gamma_i^{III} m_i/m_i^{\circ}}{\gamma_i^{II} x_i}\right] 
\]
Therefore, referring back to conventions II and III, $\gamma_i \rightarrow 1$ at \textit{infinite dilution}:
\[
\begin{aligned}
\frac{\gamma_i^{III} m_i/m_i^{\circ}}{\gamma_i^{II} x_i} & = \lim_{x_0 \to 1} \frac{\gamma_i^{III} m_i/m_i^{\circ}}{\gamma_i^{II} x_i} = \frac{1}{m_i^0} \times \lim_{x_0 \to 1} \frac{x_i}{m_i} \\
& = \frac{1}{m_i^0} \times \lim_{x_0 \to 1} \left[ \frac{n_i/ \left(\frac{n_i}{n_0 M_0}\right)}{n_0+\sum_{j \neq 0}n_j} \right] \\
& = \frac{1}{m_i^\circ M_0}
\end{aligned}
\]
This leads to the following relationship between activity coefficients calculated by each convention:
\[
\gamma_i^{II} = \gamma_i^{III} \times \frac{m_i M_0}{x_i}
\]
The standard form of this relationship is presented in \cite{Hamer1968} by assuming complete dissociation of a solute $A_0$ into $\nu$ components, so that for all solutes the sum of molalities can be summarized as $\sum_{j \neq 0}m_j = \nu m_j $:
\[
A_0 \rightleftharpoons A_1 + A_2 + A_2 + ... + A_{\nu}
\]
Other conventions used for expressing composition, using subscript 0 for solvent, and others for solutes:
\begin{tabular}{|lll|}
\hline
variable      & represented as                                                                                    & [=] units                     \\
mole number   & $n_i = w_i M_i$                                                                                   & mol i                         \\
mole fraction & $x_i = \frac{n_i}{\sum_i{n_i}} =  \frac{n_i}{n_0 + \sum_{j \neq 0}{n_i}}$                         & [adim.]                       \\
molal         & $m_i = \frac{n_i}{n_0 M_0} = \frac{n_i}{n_0 M_0 / \left(1000 \frac{g}{kg} \right)}$               & $\frac{mol_i} {kg_{solvent}}$ \\
molar         & $c_i = \frac{n_i}{V_r} = \frac{n_i}{w/\rho} = \frac{n_i \rho}{n_0 M_0 + \sum_{j\neq 0}{n_j M_j}}$ & $\frac{mol_i}{L}$\\ 
              & $= \frac{m_i \rho}{1 + \sum_{j\neq 0}{m_j M_j}}$     &       \\
\hline
\end{tabular}
 \begin{thebibliography}{9}
 \bibitem{Denbigh1968} Denbigh, Kenneth G.; The principles of chemical equilibrium; 4th ed. Cambridge University Press, UK 1981.
 \bibitem{Hamer1968} NSRDS 24 Theoretical Mean Activity Coefficients of Strong Electrolytes in Aqueous Solutions from 0 to 100oC - Walter J. Hamer. NSRDS-NBS 24, 271p. (1968). www.nist.gov/data/nsrds/NSRDS-NBS-24.pdf [Apr-2016]
 \end{thebibliography}
\end{document}
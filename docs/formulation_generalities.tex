\documentclass[12pt,a4paper,preview]{standalone}
\usepackage{amsmath}
\begin{document}
\textbf{Conventions and chemical potential of (non-ideal) solutions} \\
\textbf{Ref. [1]} \\
Convention II: Ideal solvent toward unit mole fraction; ideal solutes at \textit{infinite dilution}, expressed in mole fraction. \\
\begin{equation}
\begin{aligned}
\textnormal{solvent (0): } & (\gamma_0 \rightarrow 1) as  (x_0 \rightarrow 1)\\
\textnormal{solute (i): } & (\gamma_i \rightarrow 1) as (x_i \rightarrow 0)
\end{aligned}
\end{equation}
Convention III: Ideal solvent toward unit mole fraction; ideal solutes at \textit{infinite dilution}, expressed in molality. \\
\begin{equation}
\begin{aligned}
\textnormal{solvent (0): } & (\gamma_0 \rightarrow 1) as  (x_0 \rightarrow 1)\\
\textnormal{solute (i): } & (\gamma_i \rightarrow 1) as (m_i \rightarrow 0)
\end{aligned}
\end{equation}
Chemical potential expressed under each convention, where $m_i^{\circ}$ is a reference solute state at unit molality.
\begin{equation}
\begin{aligned}
\mu_i = & \mu_i^{*} + RTln(\gamma_i^{II} x_i) \\
	=	& \mu_i^{\fbox{}} + RTln(\gamma_i^{III} m_i/m_i^{\circ})
\end{aligned}
\end{equation}
Notes on conversion between conventions: 
\begin{itemize}
\item Values of activity coefficients $\gamma_i^{II}$ and $\gamma_i^{III}$ vary, chemical potential does not.
\item Reference states chemical potentials for each convention $\mu_i^{*}$ and $\mu_i^{\fbox{}}$  are functions of temperature and pressure exclusively, therefore their difference calculated at infinite dilution does not change with composition:
\end{itemize}
\[
\mu_i^{*} - \mu_i^{\fbox{}}  =  func(T,P) =  lim_{x_0 \rightarrow 1} \left( \mu_i^{*} - \mu_i^{\fbox{}}\right)
\]
Therefore, referring back to conventions II and III, $\gamma_i \rightarrow 1$ at \textit{infinite dilution}:
\[
\begin{aligned}
RTln\left[ \frac{\gamma_i^{III} m_i/m_i^{\circ}}{\gamma_i^{II} x_i}\right] & = lim_{x_0 \rightarrow 1} RTln\left[ \frac{\gamma_i^{III} m_i/m_i^{\circ}}{\gamma_i^{II} x_i}\right] \\
\frac{\gamma_i^{III} m_i/m_i^{\circ}}{\gamma_i^{II} x_i} & = lim_{x_0 \rightarrow 1} \frac{\gamma_i^{III} m_i/m_i^{\circ}}{\gamma_i^{II} x_i} = \frac{1}{m_i^0} \times lim_{x_0 \rightarrow 1} \frac{x_i}{m_i} \\
& = \frac{1}{m_i^0} \times lim_{x_0 \rightarrow 1} \left[ \frac{n_i/ \left(\frac{n_i}{n_0 M_0}\right)}{n_0+\sum_{j \neq 0}n_j} \right] = \frac{1}{m_i^0} \times M_0
\end{aligned}
\]
Refs. \\ 
 1 - Denbigh, Kenneth G.; The principles of chemical equilibrium; 4th ed. Cambridge University Press, UK 1981. \\ 
 2 - Hamer, Walter J.; Theoretical mean activity coefficients of strong electrolytes in aqueous solutions from 0 to 100°C. National Reference Standard Series - National Buerau of Standars 24; Issued December 1968.
\end{document}